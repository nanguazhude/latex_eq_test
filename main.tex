
\documentclass[10pt,hyperref,UTF8]{ctexbook}%使用大号字体



%解决新版latex一些BUG
\let\counterwithout\relax
\let\counterwithin\relax

%@P143
\usepackage{xcolor}        %导入包xcolor
\usepackage[
a4paper ,
left=2.5cm,                %靠近装订线的边距
right=2.2cm,               %远离装订线的边距
top=2.0cm,
bottom=1.3cm,
footskip=0.7cm,
headheight=1.3cm,
headsep=0.5cm,
marginparsep=0.6cm,        %边注与内容边距
marginparwidth=1.6cm       %边注宽度
]{geometry}                %导入包geometry

%引入边注包
\usepackage{marginnote}

%导入常用包
\usepackage{graphicx}
\usepackage{float}
\usepackage{amsmath}
\usepackage{cite}
\usepackage{caption}
\usepackage{titlesec}
\usepackage{chngcntr}
\usepackage{setspace}
\usepackage{tocbibind}  %设置目录
\usepackage{tocloft}
\usepackage{multicol}
\usepackage{listings}   %引入程序代码
\usepackage{varwidth}   %P125
\usepackage{verbatim}

%%常见符号
\usepackage{wasysym}    %
\usepackage{textcomp}   %
\usepackage{pifont}     %

%特殊符号
%对号 \checked
%问号 \textquestiondown
\usepackage{pifont}
%错号 \ding{55}
%花 \ding{96}

\usepackage{color}
\definecolor{colorbackgroundthisproject}{rgb}{1,1,1} %页面背景颜色
\definecolor{colortextthisproject}{rgb}{0,0,0}       %文字颜色
%设置页面颜色
\pagecolor{colorbackgroundthisproject}
%设置字体颜色
\color{colortextthisproject}

\usepackage{xhfill}
%\usepackage{times} do not use this pack ...

\usepackage{placeins}

%设置输出pdf格式
\usepackage[
colorlinks=true ,
%bookmarks=true,
%bookmarksopen=false,
%pdfpagemode=FullScreen,
%pdfstartview=Fit,
bookmarksnumbered=true,
pdftitle={Qml} ,       %标题
pdfauthor={Qml} ,      %作者
pdfsubject={Qml} ,     %主题
pdfkeywords={Qml} ,    %关键字
linkcolor=colortextthisproject ,
anchorcolor=colortextthisproject ,
citecolor=colortextthisproject ,
urlcolor=colortextthisproject
]{hyperref}

\usepackage{fontspec}
\definecolor{sourcegrayone}{rgb}{0.90,0.90,0.91}  %源代码背景颜色
\newfontfamily\sourcefontone{Consolas}            %源代码字体
\newfontfamily\sourcefontthree{DejaVu Sans Mono}  %源代码字体
%sudo apt-get install ttf-mscorefonts-installer   %linux安装字体
\newfontfamily\sourcefonttwo{Times New Roman}     %正文特殊符号字体

%%%%%%%%%%%%%%%%%%%%%%%%%%%%%%%%%%%%%%
\setmainfont{Times New Roman}
\setsansfont{DejaVu Sans}
\setmonofont{Latin Modern Mono}
%%%%%%%%%%%%%%%%%%%%%%%%%%%%%%%%%%%%%%

%源代码默认样式
\lstset{float,
	language=C,
	breaklines=true,
	basicstyle=     \scriptsize\sourcefontthree        , %设置字号,字体
	stringstyle=    \scriptsize\sourcefontthree        , %设置字号,字体
	keywordstyle=   \scriptsize\sourcefontthree        , %设置字号,字体
	commentstyle=   \scriptsize\sourcefontthree        , %设置字号,字体
	identifierstyle=\scriptsize\sourcefontthree        , %设置字号,字体
	numbers=left,
	numbersep=0.5em,
	numberstyle=    \scriptsize\slshape\sourcefontone  , %设置字号,字体
	frame=single,
	backgroundcolor=\color{sourcegrayone},
	showstringspaces=false ,
	aboveskip=1pt,
	belowskip=1pt,
	abovecaptionskip=1pt,
	belowcaptionskip=5pt
}
%\ifoddpage \lstset{numbers=left} \else \lstset{numbers=right}
\usepackage{changepage}    %判断奇数页偶数页

%设置item样式......
\usepackage{enumitem}
\setenumerate[1]{itemsep=0pt,partopsep=0pt,parsep=\parskip,topsep=5pt}
\setitemize[1]{itemsep=0pt,partopsep=0pt,parsep=\parskip,topsep=5pt}
\setdescription{itemsep=0pt,partopsep=0pt,parsep=\parskip,topsep=5pt}

%表
\usepackage{longtable}
\usepackage{booktabs}

%设置常量
\title{Qt Quick全面导引}                              %书籍名称
\author{Good Luck}                                   %作者名

%下划线
\usepackage{ulem}

%设置ctex
%@P135
\CTEXsetup[ number={ \arabic{chapter} } ]{chapter}
\CTEXsetup[ beforeskip={0ex},afterskip={0ex} ]{section}
\CTEXsetup[ beforeskip={0ex},afterskip={0ex} ]{subsection}
\CTEXsetup[ beforeskip={0ex},afterskip={0ex} ]{subsubsection}
%\CTEXsetup[ number={ \arabic{section} } , name={第,节} ]{section}

% Calibri
% http://www.uisdc.com/western-fonts-typesetting
\newfontfamily{\sourcefontfive}{Calibri}
\newenvironment{littlelongworld}
{  \small         \\\hspace*{\fill} \slshape\sourcefontfive   }
{                   \hspace*{\fill}\\ }

%设置路径计数器
\newcommand\treeindexnumbernameone{路径}
%\newcommand\theTreeIndexNumber{}
\newcounter{treeindexnumber}[section]
%\stepcounter{treeindexnumber}
%\refstepcounter{treeindexnumber}
\renewcommand\thetreeindexnumber{\thesection.\arabic{treeindexnumber}}

%设置命令计数器
\newcommand\commandnumbernameone{命令}
\newcounter{commandnumber}[section]
\renewcommand\thecommandnumber{\thesection.\arabic{commandnumber}}

%设置源码计数器
\newcommand\filesourcenumbernameone{源码}
\newcounter{filesourcenumber}[section]
\renewcommand\thefilesourcenumber{\thesection.\arabic{filesourcenumber}}

%设置图片计数器
\counterwithin{figure}{section}
\renewcommand\thefigure{\thesection.\arabic{figure}}

%设置表计数器
\counterwithin{table}{section}
\renewcommand\thetable{\thesection.\arabic{table}}

%%%%%%%%%%%%%%%%%%%%%%%%%%%%
%解决目录字体重叠BUG
\makeatletter
\renewcommand{\numberline}[1]{%
	\settowidth\@tempdimb{#1\hspace{0.5em}}%
	\ifdim\@tempdima<\@tempdimb%
	\@tempdima=\@tempdimb%
	\fi%
	\hb@xt@\@tempdima{\@cftbsnum #1\@cftasnum\hfil}\@cftasnumb}
\makeatother
%%%%%%%%%%%%%%%%%%%%%%%%%%%%

\lstnewenvironment{thebookfilesourceone}[1][]{
	%begin env ...
	\lstset{#1}
}{
	%end env ...
}

\lstnewenvironment{thebookfilesourceonepathtree}[1][]{
	%begin env ...
	\lstset{#1}
}{
	%end env ...
}

%命令行环境
\lstnewenvironment{thebookfilesourceonecommand}[1][]{
	%begin env ...
	\lstset{
		basicstyle=     \scriptsize\itshape\sourcefontone        , %设置字号,字体
		stringstyle=    \scriptsize\itshape\sourcefontone        , %设置字号,字体
		keywordstyle=   \scriptsize\itshape\sourcefontone        , %设置字号,字体
		commentstyle=   \scriptsize\itshape\sourcefontone        , %设置字号,字体
		identifierstyle=\scriptsize\itshape\sourcefontone        , %设置字号,字体
		#1 }
}{
	%end env ...
}

%设置标题
\setlength{\belowcaptionskip}{0.1em}
\setlength{\LTpost}{0pt}
%\setlength{\LTpre}{0pt}

\newcommand\thebookexistone{\rotatebox[origin=c]{12}{\scalebox{0.65}{$\exists$}}}
\newcommand\thebookallone{\rotatebox[origin=c]{-6}{$\forall$}}

%表格行距
\renewcommand\arraystretch{0.9}
%\setlength\belowrulesep{0pt}
%\setlength\aboverulesep{0pt}
%标题上部额外间距
\setlength{\abovecaptionskip}{5pt}
\setlength{\belowcaptionskip}{3pt}
%338
\setlength{\floatsep}{10pt plus 2pt minus 2pt}
\setlength{\textfloatsep}{10pt plus 2pt minus 2pt}
\setlength{\intextsep}{10pt plus 2pt minus 2pt}

\newcommand{\theBookRawRef}[1]{\ref{#1}}
\newcommand{\theBookLeftRightRefWrap}[1]{[#1]}
\newcommand{\refTheBookFigure}[1]{\theBookLeftRightRefWrap{\figurename\theBookRawRef{#1}}}
\newcommand{\refTheBookTable}[1]{\theBookLeftRightRefWrap{\tablename\theBookRawRef{#1}}}
\newcommand{\refTheBookFileSource}[1]{\theBookLeftRightRefWrap{\filesourcenumbernameone\theBookRawRef{#1}}}
\newcommand{\refTheBookTreeIndex}[1]{\theBookLeftRightRefWrap{\treeindexnumbernameone\theBookRawRef{#1}}}
\newcommand{\refTheBookCommand}[1]{\theBookLeftRightRefWrap{\commandnumbernameone\theBookRawRef{#1}}}
\newcommand{\refTheBookChapter}[1]{\theBookLeftRightRefWrap{第\theBookRawRef{#1}章}}
\newcommand{\refTheBookSection}[1]{\theBookLeftRightRefWrap{第\theBookRawRef{#1}节}}

\newcommand{\moduleTheBook}[1]{“\mbox{\mbox{模块}#1}”}

\newcommand{\qtVersion}{5.12.2}
\newcommand{\boostVersion}{1.69.0}
\newcommand{\ffmpegVersion}{4.1}
\newcommand{\cppVersion}{17}

\usepackage{tcolorbox}
\usepackage{boxedminipage}
%设置公式计数器
\newcommand\fileequalnumbernameone{公式}
\newcounter{fileequalnumber}[section]
\renewcommand\thefileequalnumber{\thesection.\arabic{fileequalnumber}}

\begin{document}
	
	%设置标点挤压模式
	\punctstyle{banjiao}
	
	\frontmatter
	%%%%%%%%%%%%%%%%%%%%%%%%%%%%%%%%%%%%%%%%%%%%%%%%%%%%%%%%%%%%%%%%%%%%%%%%%%%%%
	
	\pagestyle{empty}                 %关闭页眉页脚
	\maketitle                        %生成封面
	%%%%%%%%%%%%%%%%%%%%%%%%%%%%%%%%%%%%%%%%%%%%%%%%%%%%%%%%%%%%%%%%%%%%%%%%%%%%%
	
	\cleardoublepage
	\pagestyle{headings}              %开启页眉页脚
	\pagenumbering{roman}             %重新开始页码编号
	\setcounter{tocdepth}{4}          %设置目录深度
	\setcounter{secnumdepth}{4}       %设置编号深度
	\tableofcontents                  %生成目录
	%%%%%%%%%%%%%%%%%%%%%%%%%%%%%%%%%%%%%%%%%%%%%%%%%%%%%%%%%%%%%%%%%%%%%%%%%%%%%
	
	\mainmatter
	%
	%    arabic - 阿拉伯数字
	%    roman  - 小写的罗马数字
	%    Roman  - 大写的罗马数字
	%    alph   - 小写的字符形式
	%    Alph   - 大写的字符形式
	%
	\pagenumbering{arabic}           %重新开始页码编号
	%%%%%%%%%%%%%%%%%%%%%%%%%%%%%%%%%%%%%%%%%%%%%%%%%%%%%%%%%%%%%%%%%%%%%%%%%%%%%
	
    \chapter{公式测试}
    \section{测试一}
    
    %cd /D C:\texlive\2018\bin\win32
    %texdoc tcolorbox
    
     
 %\protect{\marginpar[yyy]{xxx}} 
 	 {
 	 	\setlength\abovedisplayskip{0pt}
 	 	\setlength\belowdisplayskip{0pt} 
 	 	\setlength\abovedisplayshortskip{0pt}
 	 	\setlength\belowdisplayshortskip{0pt} 
 	 	\setlength\jot{0pt}
     \centerline{\noindent\fileequalnumbernameone\ \ref{xxxewxxx}}  
     \begin{tcolorbox}[arc=0pt ,
    		boxsep=0mm ,
    	top=0pt,
    	bottom=0pt , 
    	left=0pt,
    	right=0pt,
    	leftrule=0pt,
    	rightrule=0pt,
    	toprule=0pt,
    	bottomrule=0pt,
    	titlerule=0pt,
    	toptitle=0pt,
    	bottomtitle=0pt,
    	colback=sourcegrayone,
    	colframe=sourcegrayone 
    	]\refstepcounter{fileequalnumber}\label{xxxewxxx}\noindent\begin{equation*}
a + b = 2     
\marginnote{\setlength\fboxsep{2pt}\fbox{\footnotesize{\kaishu\parbox{1em}{\setlength{\baselineskip}{2pt}\fileequalnumbernameone}}\footnotesize{\thefileequalnumber}}} \end{equation*} 
\end{tcolorbox} 
}

{
	\setlength\abovedisplayskip{0pt}
	\setlength\belowdisplayskip{0pt} 
	\setlength\abovedisplayshortskip{0pt}
	\setlength\belowdisplayshortskip{0pt} 
	\setlength\jot{1pt}

     \centerline{\noindent\fileequalnumbernameone\ \ref{xxxewxxx11}}\noindent\begin{tcolorbox}[arc=0pt ,
	boxsep=0mm ,
	top=0pt,
	bottom=0pt , 
	left=0pt,
	right=0pt,
	leftrule=0pt,
	rightrule=0pt,
	toprule=0pt,
	bottomrule=0pt,
	titlerule=0pt,
	toptitle=0pt,
	bottomtitle=0pt,
	colback=sourcegrayone,
	colframe=sourcegrayone  ]\refstepcounter{fileequalnumber}\label{xxxewxxx11}\vspace{-3pt}\begin{gather*}  
	   a + b = 2       \\
	c + d = 5      \marginnote{\setlength\fboxsep{2pt}\fbox{\footnotesize{\kaishu\parbox{1em}{\setlength{\baselineskip}{2pt}\fileequalnumbernameone}}\footnotesize{\thefileequalnumber}}}
 \end{gather*} \end{tcolorbox} 
 
}

{
	\setlength\abovedisplayskip{0pt  }
\setlength\belowdisplayskip{0pt } 
\setlength\abovedisplayshortskip{0pt}
\setlength\belowdisplayshortskip{0pt } 
\setlength\jot{0pt}
ssssssssssssssssssssssssssssssssssssssssssssssssssssssssssssssss
\begin{gather*}  
a + b = 2       \\
c + d = 5      
\end{gather*}
ssssssssssssssssssssssssssssssssssssssssssssssssssssssssssssssss

ssssssssssssssssssssssssssssssssssssssssssssssssssssssssssssssss
}

\begin{boxedminipage}{10cm} 
	\setlength\abovedisplayskip{0pt}
	\setlength\belowdisplayskip{0pt} 
	\setlength\abovedisplayshortskip{0pt}
	\setlength\belowdisplayshortskip{0pt} 
	\vspace{-5pt}\begin{gather*}a + b = 2       \\
	 c + d = 5       
	 \end{gather*} 
\end{boxedminipage}

 \framebox[ 10cm]{ text}
 \begin{gather*}  
	a + b = 2       \\
	c + d = 5       
	\end{gather*} 
 

  \noindent{aaa}
  \begin{tcolorbox}[arc=0pt ,
	boxsep=0mm ,
	top=1pt,
	bottom=1pt , 
	colback=sourcegrayone,
	colframe=sourcegrayone   
	] \noindent{aaa}   \marginnote[l]{r} 
\end{tcolorbox} 

\marginnote{\noindent{xxx}} 
 
 

   {
	
	
	\centerline{\noindent\fileequalnumbernameone\ \ref{xxxewxxx1}}  
	\begin{tcolorbox}[arc=0pt ,
		boxsep=0mm ,
		top=1pt,
		bottom=1pt , 
		colback=sourcegrayone,
		colframe=sourcegrayone   
		] \refstepcounter{fileequalnumber}\label{xxxewxxx1} \noindent\begin{equation*}
		a + b = 2\end{equation*}\end{tcolorbox}
}
    
    {
    	\tcbset{ arc=0mm    }
    	
    	\begin{tcolorbox}\noindent\begin{equation*}
    		a + b = 2  \end{equation*}\end{tcolorbox}
    }
    
 
	
	\backmatter
	%%%%%%%%%%%%%%%%%%%%%%%%%%%%%%%%%%%%%%%%%%%%%%%%%%%%%%%%%%%%%%%%%%%%%%%%%%%%%
	
\end{document}



